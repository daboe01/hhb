\documentclass[a4]{scrreprt}
\usepackage[ngerman]{babel}
\usepackage[latin1]{inputenc}
\usepackage{<copytex:hhb/formular.sty>}

\renewcommand{\familydefault}{\sfdefault} 
\usepackage{helvet} 

\usepackage{<copytex:hhb/wallpaper.sty>}
\setlength{\headheight}{14\baselineskip}
\CenterWallPaper{1}{<file:briefkopfHHB2.pdf>}

\newFRMfield{Name}{15mm}
\newFRMfield{Vorname}{15mm}
\newFRMfield{date}{15mm}
\newFRMfield{time}{15mm}
\setFRMinlinestyle 
\newFRMenvironment{env2}{Todeszeichen}{3}
\newFRMfield{ort}{10mm}[Ort][Freiburg] 
\setFRMruledstyle 
\newFRMfield{sigf}{50mm}[Unterschrift]

\begin{document}

\setcounter{secnumdepth}{0}
\section{Best"atigung}

\noindent
Bei Frau/Herrn \useFRMfield{Name}[<var:Name>],  \useFRMfield{Vorname}[<var:Vorname>],\useFRMfield{date}[<var:Geburtsdatum>] habe ich am \useFRMfield{date} um \useFRMfield{time} Uhr als "au"seres sicheres \\ \\
\begin{env2} \end{env2} 
festgestellt. \\ \\

\noindent
Damit sind der Tod und auch der endg"ultige, nicht behebbare Ausfall der Gesamtfunktion des Gehirns nachgewiesen. \\ \\

\noindent
Freiburg, <var:Datum>, \hfill \useFRMfield{sigf}

\end{document}
